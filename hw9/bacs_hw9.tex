% Options for packages loaded elsewhere
\PassOptionsToPackage{unicode}{hyperref}
\PassOptionsToPackage{hyphens}{url}
%
\documentclass[
]{article}
\usepackage{amsmath,amssymb}
\usepackage{iftex}
\ifPDFTeX
  \usepackage[T1]{fontenc}
  \usepackage[utf8]{inputenc}
  \usepackage{textcomp} % provide euro and other symbols
\else % if luatex or xetex
  \usepackage{unicode-math} % this also loads fontspec
  \defaultfontfeatures{Scale=MatchLowercase}
  \defaultfontfeatures[\rmfamily]{Ligatures=TeX,Scale=1}
\fi
\usepackage{lmodern}
\ifPDFTeX\else
  % xetex/luatex font selection
\fi
% Use upquote if available, for straight quotes in verbatim environments
\IfFileExists{upquote.sty}{\usepackage{upquote}}{}
\IfFileExists{microtype.sty}{% use microtype if available
  \usepackage[]{microtype}
  \UseMicrotypeSet[protrusion]{basicmath} % disable protrusion for tt fonts
}{}
\makeatletter
\@ifundefined{KOMAClassName}{% if non-KOMA class
  \IfFileExists{parskip.sty}{%
    \usepackage{parskip}
  }{% else
    \setlength{\parindent}{0pt}
    \setlength{\parskip}{6pt plus 2pt minus 1pt}}
}{% if KOMA class
  \KOMAoptions{parskip=half}}
\makeatother
\usepackage{xcolor}
\usepackage[margin=1in]{geometry}
\usepackage{color}
\usepackage{fancyvrb}
\newcommand{\VerbBar}{|}
\newcommand{\VERB}{\Verb[commandchars=\\\{\}]}
\DefineVerbatimEnvironment{Highlighting}{Verbatim}{commandchars=\\\{\}}
% Add ',fontsize=\small' for more characters per line
\usepackage{framed}
\definecolor{shadecolor}{RGB}{248,248,248}
\newenvironment{Shaded}{\begin{snugshade}}{\end{snugshade}}
\newcommand{\AlertTok}[1]{\textcolor[rgb]{0.94,0.16,0.16}{#1}}
\newcommand{\AnnotationTok}[1]{\textcolor[rgb]{0.56,0.35,0.01}{\textbf{\textit{#1}}}}
\newcommand{\AttributeTok}[1]{\textcolor[rgb]{0.13,0.29,0.53}{#1}}
\newcommand{\BaseNTok}[1]{\textcolor[rgb]{0.00,0.00,0.81}{#1}}
\newcommand{\BuiltInTok}[1]{#1}
\newcommand{\CharTok}[1]{\textcolor[rgb]{0.31,0.60,0.02}{#1}}
\newcommand{\CommentTok}[1]{\textcolor[rgb]{0.56,0.35,0.01}{\textit{#1}}}
\newcommand{\CommentVarTok}[1]{\textcolor[rgb]{0.56,0.35,0.01}{\textbf{\textit{#1}}}}
\newcommand{\ConstantTok}[1]{\textcolor[rgb]{0.56,0.35,0.01}{#1}}
\newcommand{\ControlFlowTok}[1]{\textcolor[rgb]{0.13,0.29,0.53}{\textbf{#1}}}
\newcommand{\DataTypeTok}[1]{\textcolor[rgb]{0.13,0.29,0.53}{#1}}
\newcommand{\DecValTok}[1]{\textcolor[rgb]{0.00,0.00,0.81}{#1}}
\newcommand{\DocumentationTok}[1]{\textcolor[rgb]{0.56,0.35,0.01}{\textbf{\textit{#1}}}}
\newcommand{\ErrorTok}[1]{\textcolor[rgb]{0.64,0.00,0.00}{\textbf{#1}}}
\newcommand{\ExtensionTok}[1]{#1}
\newcommand{\FloatTok}[1]{\textcolor[rgb]{0.00,0.00,0.81}{#1}}
\newcommand{\FunctionTok}[1]{\textcolor[rgb]{0.13,0.29,0.53}{\textbf{#1}}}
\newcommand{\ImportTok}[1]{#1}
\newcommand{\InformationTok}[1]{\textcolor[rgb]{0.56,0.35,0.01}{\textbf{\textit{#1}}}}
\newcommand{\KeywordTok}[1]{\textcolor[rgb]{0.13,0.29,0.53}{\textbf{#1}}}
\newcommand{\NormalTok}[1]{#1}
\newcommand{\OperatorTok}[1]{\textcolor[rgb]{0.81,0.36,0.00}{\textbf{#1}}}
\newcommand{\OtherTok}[1]{\textcolor[rgb]{0.56,0.35,0.01}{#1}}
\newcommand{\PreprocessorTok}[1]{\textcolor[rgb]{0.56,0.35,0.01}{\textit{#1}}}
\newcommand{\RegionMarkerTok}[1]{#1}
\newcommand{\SpecialCharTok}[1]{\textcolor[rgb]{0.81,0.36,0.00}{\textbf{#1}}}
\newcommand{\SpecialStringTok}[1]{\textcolor[rgb]{0.31,0.60,0.02}{#1}}
\newcommand{\StringTok}[1]{\textcolor[rgb]{0.31,0.60,0.02}{#1}}
\newcommand{\VariableTok}[1]{\textcolor[rgb]{0.00,0.00,0.00}{#1}}
\newcommand{\VerbatimStringTok}[1]{\textcolor[rgb]{0.31,0.60,0.02}{#1}}
\newcommand{\WarningTok}[1]{\textcolor[rgb]{0.56,0.35,0.01}{\textbf{\textit{#1}}}}
\usepackage{graphicx}
\makeatletter
\def\maxwidth{\ifdim\Gin@nat@width>\linewidth\linewidth\else\Gin@nat@width\fi}
\def\maxheight{\ifdim\Gin@nat@height>\textheight\textheight\else\Gin@nat@height\fi}
\makeatother
% Scale images if necessary, so that they will not overflow the page
% margins by default, and it is still possible to overwrite the defaults
% using explicit options in \includegraphics[width, height, ...]{}
\setkeys{Gin}{width=\maxwidth,height=\maxheight,keepaspectratio}
% Set default figure placement to htbp
\makeatletter
\def\fps@figure{htbp}
\makeatother
\usepackage{soul}
\setlength{\emergencystretch}{3em} % prevent overfull lines
\providecommand{\tightlist}{%
  \setlength{\itemsep}{0pt}\setlength{\parskip}{0pt}}
\setcounter{secnumdepth}{-\maxdimen} % remove section numbering
\ifLuaTeX
  \usepackage{selnolig}  % disable illegal ligatures
\fi
\IfFileExists{bookmark.sty}{\usepackage{bookmark}}{\usepackage{hyperref}}
\IfFileExists{xurl.sty}{\usepackage{xurl}}{} % add URL line breaks if available
\urlstyle{same}
\hypersetup{
  pdftitle={bacs\_hw9},
  hidelinks,
  pdfcreator={LaTeX via pandoc}}

\title{bacs\_hw9}
\author{}
\date{\vspace{-2.5em}2024-04-21}

\begin{document}
\maketitle

110034002 has helped me with

\hypertarget{question-1}{%
\section{Question 1}\label{question-1}}

Let\textquotesingle s deal with \textbf{non-linearity} first. Create a
new dataset that log-transforms several variables from our original
dataset (called cars in this case)

\begin{Shaded}
\begin{Highlighting}[]
\NormalTok{cars }\OtherTok{\textless{}{-}} \FunctionTok{read.table}\NormalTok{(}\StringTok{"auto{-}data.txt"}\NormalTok{, }\AttributeTok{header =} \ConstantTok{FALSE}\NormalTok{, }\AttributeTok{na.strings =} \StringTok{"?"}\NormalTok{)}
\FunctionTok{names}\NormalTok{(cars) }\OtherTok{\textless{}{-}} \FunctionTok{c}\NormalTok{(}\StringTok{"mpg"}\NormalTok{, }\StringTok{"cylinders"}\NormalTok{, }\StringTok{"displacement"}\NormalTok{, }\StringTok{"horsepower"}\NormalTok{, }\StringTok{"weight"}\NormalTok{, }
                 \StringTok{"acceleration"}\NormalTok{, }\StringTok{"model\_year"}\NormalTok{, }\StringTok{"origin"}\NormalTok{, }\StringTok{"car\_name"}\NormalTok{)}

\NormalTok{cars\_log }\OtherTok{\textless{}{-}} \FunctionTok{with}\NormalTok{(cars, }\FunctionTok{data.frame}\NormalTok{(}\FunctionTok{log}\NormalTok{(mpg), }\FunctionTok{log}\NormalTok{(cylinders), }\FunctionTok{log}\NormalTok{(displacement), }
                                  \FunctionTok{log}\NormalTok{(horsepower), }\FunctionTok{log}\NormalTok{(weight), }\FunctionTok{log}\NormalTok{(acceleration), }
\NormalTok{                                  model\_year, origin))}
\end{Highlighting}
\end{Shaded}

\hypertarget{a}{%
\subsection{1a}\label{a}}

Run a new regression on the cars\_log dataset, with mpg.log. dependent
on all other variables

\hypertarget{i}{%
\subsubsection{i}\label{i}}

\textbf{Question}

Which log-transformed factors have a significant effect on log.mpg. at
10\% significance?

\textbf{Answer}

Acceleration, weight, model year, factor(origin), and horsepower

\begin{Shaded}
\begin{Highlighting}[]
\NormalTok{log\_regr }\OtherTok{\textless{}{-}} \FunctionTok{summary}\NormalTok{(}\FunctionTok{lm}\NormalTok{(}\AttributeTok{formula =}\NormalTok{ log.mpg. }\SpecialCharTok{\textasciitilde{}}\NormalTok{ log.cylinders. }\SpecialCharTok{+}\NormalTok{ log.displacement. }\SpecialCharTok{+} 
\NormalTok{    log.horsepower. }\SpecialCharTok{+}\NormalTok{ log.weight. }\SpecialCharTok{+}\NormalTok{ log.acceleration. }\SpecialCharTok{+}\NormalTok{ model\_year }\SpecialCharTok{+} 
    \FunctionTok{factor}\NormalTok{(origin), }\AttributeTok{data =}\NormalTok{ cars\_log, }\AttributeTok{na.action =}\NormalTok{ na.exclude))}
\NormalTok{log\_regr}
\end{Highlighting}
\end{Shaded}

\begin{verbatim}
## 
## Call:
## lm(formula = log.mpg. ~ log.cylinders. + log.displacement. + 
##     log.horsepower. + log.weight. + log.acceleration. + model_year + 
##     factor(origin), data = cars_log, na.action = na.exclude)
## 
## Residuals:
##      Min       1Q   Median       3Q      Max 
## -0.39727 -0.06880  0.00450  0.06356  0.38542 
## 
## Coefficients:
##                    Estimate Std. Error t value Pr(>|t|)    
## (Intercept)        7.301938   0.361777  20.184  < 2e-16 ***
## log.cylinders.    -0.081915   0.061116  -1.340  0.18094    
## log.displacement.  0.020387   0.058369   0.349  0.72707    
## log.horsepower.   -0.284751   0.057945  -4.914 1.32e-06 ***
## log.weight.       -0.592955   0.085165  -6.962 1.46e-11 ***
## log.acceleration. -0.169673   0.059649  -2.845  0.00469 ** 
## model_year         0.030239   0.001771  17.078  < 2e-16 ***
## factor(origin)2    0.050717   0.020920   2.424  0.01580 *  
## factor(origin)3    0.047215   0.020622   2.290  0.02259 *  
## ---
## Signif. codes:  0 '***' 0.001 '**' 0.01 '*' 0.05 '.' 0.1 ' ' 1
## 
## Residual standard error: 0.113 on 383 degrees of freedom
##   (因為不存在,6 個觀察量被刪除了)
## Multiple R-squared:  0.8919, Adjusted R-squared:  0.8897 
## F-statistic:   395 on 8 and 383 DF,  p-value: < 2.2e-16
\end{verbatim}

\hypertarget{ii}{%
\subsubsection{ii}\label{ii}}

\textbf{Question}

Do some new factors now have effects on mpg, and why might this be?

\textbf{Answer}

Acceleration and horsepower are new factors. This may arise because the
original data could be skewed, and now log-transformed to display the
otherwise hidden patterns.

\hypertarget{iii}{%
\subsubsection{iii}\label{iii}}

\textbf{Question}

Which factors still have insignificant or opposite (from correlation)
effects on mpg? Why might this be?

\textbf{Answer}

Cylinders and displacement. This may arise due to the data's
multicollinearity.

\hypertarget{b}{%
\subsection{1b}\label{b}}

\hypertarget{i-1}{%
\subsubsection{i}\label{i-1}}

Create a regression (call it regr\_wt) of mpg over weight \ul{from the
original cars dataset}

\begin{Shaded}
\begin{Highlighting}[]
\NormalTok{regr\_wt }\OtherTok{\textless{}{-}} \FunctionTok{summary}\NormalTok{(}\FunctionTok{lm}\NormalTok{(mpg }\SpecialCharTok{\textasciitilde{}}\NormalTok{ weight, }\AttributeTok{data=}\NormalTok{cars))}
\NormalTok{regr\_wt}
\end{Highlighting}
\end{Shaded}

\begin{verbatim}
## 
## Call:
## lm(formula = mpg ~ weight, data = cars)
## 
## Residuals:
##     Min      1Q  Median      3Q     Max 
## -12.012  -2.801  -0.351   2.114  16.480 
## 
## Coefficients:
##               Estimate Std. Error t value Pr(>|t|)    
## (Intercept) 46.3173644  0.7952452   58.24   <2e-16 ***
## weight      -0.0076766  0.0002575  -29.81   <2e-16 ***
## ---
## Signif. codes:  0 '***' 0.001 '**' 0.01 '*' 0.05 '.' 0.1 ' ' 1
## 
## Residual standard error: 4.345 on 396 degrees of freedom
## Multiple R-squared:  0.6918, Adjusted R-squared:  0.691 
## F-statistic: 888.9 on 1 and 396 DF,  p-value: < 2.2e-16
\end{verbatim}

\hypertarget{ii-1}{%
\subsubsection{ii}\label{ii-1}}

Create a regression (call it regr\_wt\_log) of log.mpg. on log.weight.
from cars\_log

\begin{Shaded}
\begin{Highlighting}[]
\NormalTok{regr\_wt\_log }\OtherTok{\textless{}{-}} \FunctionTok{summary}\NormalTok{(}\FunctionTok{lm}\NormalTok{(log.mpg. }\SpecialCharTok{\textasciitilde{}}\NormalTok{ log.weight., }\AttributeTok{data=}\NormalTok{cars\_log, }\AttributeTok{na.action=}\NormalTok{na.exclude))}
\NormalTok{regr\_wt\_log}
\end{Highlighting}
\end{Shaded}

\begin{verbatim}
## 
## Call:
## lm(formula = log.mpg. ~ log.weight., data = cars_log, na.action = na.exclude)
## 
## Residuals:
##      Min       1Q   Median       3Q      Max 
## -0.52408 -0.10441 -0.00805  0.10165  0.59384 
## 
## Coefficients:
##             Estimate Std. Error t value Pr(>|t|)    
## (Intercept)  11.5219     0.2349   49.06   <2e-16 ***
## log.weight.  -1.0583     0.0295  -35.87   <2e-16 ***
## ---
## Signif. codes:  0 '***' 0.001 '**' 0.01 '*' 0.05 '.' 0.1 ' ' 1
## 
## Residual standard error: 0.165 on 396 degrees of freedom
## Multiple R-squared:  0.7647, Adjusted R-squared:  0.7641 
## F-statistic:  1287 on 1 and 396 DF,  p-value: < 2.2e-16
\end{verbatim}

\hypertarget{iii-1}{%
\subsubsection{iii}\label{iii-1}}

Visualize the residuals of both regression models (raw and
log-transformed)

\hypertarget{density-plot-of-residuals}{%
\paragraph{(1) density plot of
residuals}\label{density-plot-of-residuals}}

\begin{Shaded}
\begin{Highlighting}[]
\FunctionTok{par}\NormalTok{(}\AttributeTok{mfrow =} \FunctionTok{c}\NormalTok{(}\DecValTok{1}\NormalTok{,}\DecValTok{2}\NormalTok{))}
\FunctionTok{plot}\NormalTok{(}\FunctionTok{density}\NormalTok{(regr\_wt}\SpecialCharTok{$}\NormalTok{residuals), }\AttributeTok{lwd =} \DecValTok{2}\NormalTok{, }\AttributeTok{col =} \StringTok{"cornflowerblue"}\NormalTok{, }\AttributeTok{main =} \StringTok{"Density Plot (raw)"}\NormalTok{)}
\FunctionTok{abline}\NormalTok{(}\AttributeTok{v=} \FunctionTok{mean}\NormalTok{(regr\_wt}\SpecialCharTok{$}\NormalTok{residuals))}
\FunctionTok{plot}\NormalTok{(}\FunctionTok{density}\NormalTok{(regr\_wt\_log}\SpecialCharTok{$}\NormalTok{residuals), }\AttributeTok{lwd =} \DecValTok{2}\NormalTok{, }\AttributeTok{col =} \StringTok{"coral3"}\NormalTok{, }\AttributeTok{main =} \StringTok{"Density Plot (log{-}transformed)"}\NormalTok{)}
\FunctionTok{abline}\NormalTok{(}\AttributeTok{v=} \FunctionTok{mean}\NormalTok{(regr\_wt\_log}\SpecialCharTok{$}\NormalTok{residuals))}
\end{Highlighting}
\end{Shaded}

\includegraphics{bacs_hw9_files/figure-latex/unnamed-chunk-5-1.pdf}

\hypertarget{scatterplot-of-log.weight.-vs.-residuals}{%
\paragraph{(2) scatterplot of log.weight.
vs.~residuals}\label{scatterplot-of-log.weight.-vs.-residuals}}

\begin{Shaded}
\begin{Highlighting}[]
\FunctionTok{par}\NormalTok{(}\AttributeTok{mfrow =} \FunctionTok{c}\NormalTok{(}\DecValTok{1}\NormalTok{,}\DecValTok{2}\NormalTok{))}
\FunctionTok{plot}\NormalTok{(cars\_log}\SpecialCharTok{$}\NormalTok{log.weight., regr\_wt}\SpecialCharTok{$}\NormalTok{residuals, }\AttributeTok{pch=}\DecValTok{19}\NormalTok{,}\AttributeTok{main =} \StringTok{"Density Plot (raw)"}\NormalTok{)}
\FunctionTok{plot}\NormalTok{(cars\_log}\SpecialCharTok{$}\NormalTok{log.weight., regr\_wt\_log}\SpecialCharTok{$}\NormalTok{residuals, }\AttributeTok{pch=}\DecValTok{19}\NormalTok{, }\AttributeTok{main =} \StringTok{"Density Plot (log{-}transformed)"}\NormalTok{)}
\end{Highlighting}
\end{Shaded}

\includegraphics{bacs_hw9_files/figure-latex/unnamed-chunk-6-1.pdf}

\hypertarget{iv}{%
\subsubsection{iv}\label{iv}}

\textbf{Question}

Which regression produces better distributed residuals for the
assumptions of regression?

\textbf{Answer}

The log-transformed regression.

\hypertarget{v}{%
\subsubsection{v}\label{v}}

\textbf{Question}

How would you interpret the slope of log.weight. vs log.mpg. in simple
words?

\textbf{Answer}

Based on the summary tables in (i) and (ii), it is clear that one
percent change in log.weight. leads to -1.0583 percent change in
log.mpg.

\hypertarget{vi}{%
\subsubsection{vi}\label{vi}}

\textbf{Question}

From its standard error, what is the 95\% confidence interval of the
slope of log.weight. vs log.mpg.?

\textbf{Answer}

\begin{Shaded}
\begin{Highlighting}[]
\NormalTok{regr\_wt\_log}
\end{Highlighting}
\end{Shaded}

\begin{verbatim}
## 
## Call:
## lm(formula = log.mpg. ~ log.weight., data = cars_log, na.action = na.exclude)
## 
## Residuals:
##      Min       1Q   Median       3Q      Max 
## -0.52408 -0.10441 -0.00805  0.10165  0.59384 
## 
## Coefficients:
##             Estimate Std. Error t value Pr(>|t|)    
## (Intercept)  11.5219     0.2349   49.06   <2e-16 ***
## log.weight.  -1.0583     0.0295  -35.87   <2e-16 ***
## ---
## Signif. codes:  0 '***' 0.001 '**' 0.01 '*' 0.05 '.' 0.1 ' ' 1
## 
## Residual standard error: 0.165 on 396 degrees of freedom
## Multiple R-squared:  0.7647, Adjusted R-squared:  0.7641 
## F-statistic:  1287 on 1 and 396 DF,  p-value: < 2.2e-16
\end{verbatim}

\begin{Shaded}
\begin{Highlighting}[]
\NormalTok{slope\_estimate }\OtherTok{\textless{}{-}}\NormalTok{ regr\_wt\_log}\SpecialCharTok{$}\NormalTok{coefficients[}\StringTok{"log.weight."}\NormalTok{, }\StringTok{"Estimate"}\NormalTok{]}
\NormalTok{slope\_se }\OtherTok{\textless{}{-}}\NormalTok{ regr\_wt\_log}\SpecialCharTok{$}\NormalTok{coefficients[}\StringTok{"log.weight."}\NormalTok{, }\StringTok{"Std. Error"}\NormalTok{]}

\NormalTok{CI\_lower }\OtherTok{\textless{}{-}}\NormalTok{ slope\_estimate }\SpecialCharTok{{-}} \FloatTok{1.96} \SpecialCharTok{*}\NormalTok{ slope\_se}
\NormalTok{CI\_upper }\OtherTok{\textless{}{-}}\NormalTok{ slope\_estimate }\SpecialCharTok{+} \FloatTok{1.96} \SpecialCharTok{*}\NormalTok{ slope\_se}

\FunctionTok{cat}\NormalTok{(}\StringTok{"CI\_upperbound:"}\NormalTok{, CI\_upper, }\StringTok{"}\SpecialCharTok{\textbackslash{}n}\StringTok{"}\NormalTok{)}
\end{Highlighting}
\end{Shaded}

\begin{verbatim}
## CI_upperbound: -1.000448
\end{verbatim}

\begin{Shaded}
\begin{Highlighting}[]
\FunctionTok{cat}\NormalTok{(}\StringTok{"CI\_lowerbound:"}\NormalTok{, CI\_lower)}
\end{Highlighting}
\end{Shaded}

\begin{verbatim}
## CI_lowerbound: -1.116088
\end{verbatim}

\hypertarget{question-2}{%
\section{Question 2}\label{question-2}}

Let\textquotesingle s tackle \textbf{multicollinearity} next. Consider
the regression model:

regr\_log \textless- lm(log.mpg. \textasciitilde{} log.cylinders. +
log.displacement. + log.horsepower. +

log.weight. + log.acceleration. + model\_year +

factor(origin), data=cars\_log)

\hypertarget{a-1}{%
\subsection{2a}\label{a-1}}

Using regression and R2, compute the VIF of log.weight. using the
approach shown in class

\begin{Shaded}
\begin{Highlighting}[]
\NormalTok{logweight\_regr }\OtherTok{\textless{}{-}} \FunctionTok{lm}\NormalTok{(log.weight.}\SpecialCharTok{\textasciitilde{}}\NormalTok{log.cylinders.}\SpecialCharTok{+}\NormalTok{log.displacement.}\SpecialCharTok{+}\NormalTok{log.horsepower.}\SpecialCharTok{+}\NormalTok{log.acceleration.}\SpecialCharTok{+}\NormalTok{model\_year, }\AttributeTok{data=}\NormalTok{cars\_log, }\AttributeTok{na.action =}\NormalTok{ na.exclude)}
\NormalTok{r2\_logweight\_regr }\OtherTok{\textless{}{-}} \FunctionTok{summary}\NormalTok{(logweight\_regr)}\SpecialCharTok{$}\NormalTok{r.squared}
\NormalTok{vif\_logweight }\OtherTok{\textless{}{-}} \DecValTok{1} \SpecialCharTok{/}\NormalTok{ (}\DecValTok{1} \SpecialCharTok{{-}}\NormalTok{ r2\_logweight\_regr)}
\NormalTok{vif\_logweight}
\end{Highlighting}
\end{Shaded}

\begin{verbatim}
## [1] 16.07917
\end{verbatim}

\hypertarget{b-1}{%
\subsection{2b}\label{b-1}}

Let\textquotesingle s try a procedure called Stepwise VIF Selection to
remove highly collinear predictors.

\hypertarget{i-2}{%
\subsubsection{i}\label{i-2}}

Use vif(regr\_log) to compute VIF of the all the independent variables

\begin{Shaded}
\begin{Highlighting}[]
\CommentTok{\#install.packages(\textquotesingle{}car\textquotesingle{})}
\FunctionTok{library}\NormalTok{(}\StringTok{\textquotesingle{}car\textquotesingle{}}\NormalTok{)}
\end{Highlighting}
\end{Shaded}

\begin{verbatim}
## 載入需要的套件:carData
\end{verbatim}

\begin{Shaded}
\begin{Highlighting}[]
\NormalTok{regr\_log }\OtherTok{\textless{}{-}} \FunctionTok{lm}\NormalTok{(log.weight. }\SpecialCharTok{\textasciitilde{}}\NormalTok{ log.cylinders.}\SpecialCharTok{+}\NormalTok{log.displacement.}\SpecialCharTok{+}\NormalTok{log.horsepower.}\SpecialCharTok{+}\NormalTok{log.acceleration.}\SpecialCharTok{+}\NormalTok{model\_year,}
                 \AttributeTok{data=}\NormalTok{cars\_log, }\AttributeTok{na.action=}\NormalTok{na.exclude)}
\FunctionTok{vif}\NormalTok{(regr\_log)}
\end{Highlighting}
\end{Shaded}

\begin{verbatim}
##    log.cylinders. log.displacement.   log.horsepower. log.acceleration. 
##          9.748860         13.412802          7.013535          2.253283 
##        model_year 
##          1.198164
\end{verbatim}

\hypertarget{ii-2}{%
\subsubsection{ii}\label{ii-2}}

Eliminate from your model the single independent variable with the
largest VIF score that is also greater than 5

\begin{Shaded}
\begin{Highlighting}[]
\CommentTok{\# log.displacement scores the largest VIF}
\NormalTok{regr\_log }\OtherTok{\textless{}{-}} \FunctionTok{lm}\NormalTok{(log.weight. }\SpecialCharTok{\textasciitilde{}}\NormalTok{ log.cylinders.}\SpecialCharTok{+}\NormalTok{log.horsepower.}\SpecialCharTok{+}\NormalTok{log.acceleration.}\SpecialCharTok{+}\NormalTok{model\_year,}
                 \AttributeTok{data=}\NormalTok{cars\_log, }\AttributeTok{na.action=}\NormalTok{na.exclude)}
\FunctionTok{vif}\NormalTok{(regr\_log)}
\end{Highlighting}
\end{Shaded}

\begin{verbatim}
##    log.cylinders.   log.horsepower. log.acceleration.        model_year 
##          3.326803          5.208472          2.167932          1.190458
\end{verbatim}

\hypertarget{iii-2}{%
\subsubsection{iii}\label{iii-2}}

Repeat steps (i) and (ii) until no more independent variables have VIF
scores above 5

\begin{Shaded}
\begin{Highlighting}[]
\CommentTok{\# Keep on to eliminate log.horsepower, whose VIF is at once the largest and greater than 5}
\NormalTok{regr\_log }\OtherTok{\textless{}{-}} \FunctionTok{lm}\NormalTok{(log.weight. }\SpecialCharTok{\textasciitilde{}}\NormalTok{ log.cylinders.}\SpecialCharTok{+}\NormalTok{log.acceleration.}\SpecialCharTok{+}\NormalTok{model\_year,}
                 \AttributeTok{data=}\NormalTok{cars\_log, }\AttributeTok{na.action=}\NormalTok{na.exclude)}
\FunctionTok{vif}\NormalTok{(regr\_log)}
\end{Highlighting}
\end{Shaded}

\begin{verbatim}
##    log.cylinders. log.acceleration.        model_year 
##          1.412557          1.382461          1.165129
\end{verbatim}

\hypertarget{iv-1}{%
\subsubsection{iv}\label{iv-1}}

Report the final regression model and its summary statistics

\begin{Shaded}
\begin{Highlighting}[]
\FunctionTok{summary}\NormalTok{(regr\_log)}
\end{Highlighting}
\end{Shaded}

\begin{verbatim}
## 
## Call:
## lm(formula = log.weight. ~ log.cylinders. + log.acceleration. + 
##     model_year, data = cars_log, na.action = na.exclude)
## 
## Residuals:
##      Min       1Q   Median       3Q      Max 
## -0.35101 -0.09406 -0.00256  0.09311  0.41564 
## 
## Coefficients:
##                   Estimate Std. Error t value Pr(>|t|)    
## (Intercept)       6.398532   0.198541  32.228   <2e-16 ***
## log.cylinders.    0.835451   0.026327  31.734   <2e-16 ***
## log.acceleration. 0.035708   0.043451   0.822    0.412    
## model_year        0.001084   0.001950   0.556    0.579    
## ---
## Signif. codes:  0 '***' 0.001 '**' 0.01 '*' 0.05 '.' 0.1 ' ' 1
## 
## Residual standard error: 0.1331 on 394 degrees of freedom
## Multiple R-squared:  0.7769, Adjusted R-squared:  0.7752 
## F-statistic: 457.3 on 3 and 394 DF,  p-value: < 2.2e-16
\end{verbatim}

\hypertarget{c}{%
\subsection{2c}\label{c}}

\textbf{Question}

Using stepwise VIF selection, have we lost any variables that were
previously significant? If so, how much did we hurt our explanation by
dropping those variables?

\textbf{Answer}

We eliminated displacement and horsepower, which used to seem
significant. In dropping these variables, the explanation of the fit
model can be hurt due to the R-squared change.

\hypertarget{d}{%
\subsection{2d}\label{d}}

From only the formula for VIF, try deducing/deriving the following:

\hypertarget{i-3}{%
\subsubsection{i}\label{i-3}}

\textbf{Question}

If an independent variable has no correlation with other independent
variables, what would its VIF score be?

\textbf{Answer}

By the VIF formula 1 / (1 - R-squared), no correlation means R-squared
to be 0, and VIF in turn becomes 1.

\hypertarget{ii-3}{%
\subsubsection{ii}\label{ii-3}}

\textbf{Question}

Given a regression with only two independent variables (X1 and X2), how
correlated would X1 and X2 have to be, to get VIF scores of 5 or higher?
To get VIF scores of 10 or higher?

\textbf{Answer}

Correlation would have to be above 0.894 to get VIF scores of 5 or
higher.

Correlation would have to be above 0.948 to get VIF scores of 10 or
higher.

\hypertarget{question-3}{%
\section{Question 3}\label{question-3}}

Might the relationship of weight on mpg be different for cars from
different origins? Let\textquotesingle s try visualizing this. First,
plot all the weights, using different colors and symbols for the three
origins

\hypertarget{a-2}{%
\subsection{3a}\label{a-2}}

Let\textquotesingle s add three separate regression lines on the
scatterplot, one for each of the origins. Here\textquotesingle s one for
the US to get you started:

\begin{Shaded}
\begin{Highlighting}[]
\NormalTok{origin\_colors }\OtherTok{=} \FunctionTok{c}\NormalTok{(}\StringTok{"blue"}\NormalTok{, }\StringTok{"darkgreen"}\NormalTok{, }\StringTok{"red"}\NormalTok{)}
\FunctionTok{with}\NormalTok{(cars\_log, }\FunctionTok{plot}\NormalTok{(log.weight., log.mpg., }\AttributeTok{pch=}\NormalTok{origin, }\AttributeTok{col=}\NormalTok{origin\_colors[origin]))}

\NormalTok{cars\_us }\OtherTok{\textless{}{-}} \FunctionTok{subset}\NormalTok{(cars\_log, origin }\SpecialCharTok{==} \DecValTok{1}\NormalTok{)}
\NormalTok{wt\_regr\_us }\OtherTok{\textless{}{-}} \FunctionTok{lm}\NormalTok{(log.mpg. }\SpecialCharTok{\textasciitilde{}}\NormalTok{ log.weight., }\AttributeTok{data=}\NormalTok{cars\_us)}
\FunctionTok{abline}\NormalTok{(wt\_regr\_us, }\AttributeTok{col=}\NormalTok{origin\_colors[}\DecValTok{1}\NormalTok{], }\AttributeTok{lwd=}\DecValTok{2}\NormalTok{)}

\NormalTok{cars\_jp }\OtherTok{\textless{}{-}} \FunctionTok{subset}\NormalTok{(cars\_log, origin }\SpecialCharTok{==} \DecValTok{2}\NormalTok{)}
\NormalTok{wt\_regr\_jp }\OtherTok{\textless{}{-}} \FunctionTok{lm}\NormalTok{(log.mpg. }\SpecialCharTok{\textasciitilde{}}\NormalTok{ log.weight., }\AttributeTok{data=}\NormalTok{cars\_jp)}
\FunctionTok{abline}\NormalTok{(wt\_regr\_jp, }\AttributeTok{col=}\NormalTok{origin\_colors[}\DecValTok{2}\NormalTok{], }\AttributeTok{lwd=}\DecValTok{2}\NormalTok{)}

\NormalTok{cars\_eu }\OtherTok{\textless{}{-}} \FunctionTok{subset}\NormalTok{(cars\_log, origin }\SpecialCharTok{==} \DecValTok{3}\NormalTok{)}
\NormalTok{wt\_regr\_eu }\OtherTok{\textless{}{-}} \FunctionTok{lm}\NormalTok{(log.mpg. }\SpecialCharTok{\textasciitilde{}}\NormalTok{ log.weight., }\AttributeTok{data=}\NormalTok{cars\_eu)}
\FunctionTok{abline}\NormalTok{(wt\_regr\_eu, }\AttributeTok{col=}\NormalTok{origin\_colors[}\DecValTok{3}\NormalTok{], }\AttributeTok{lwd=}\DecValTok{2}\NormalTok{)}
\end{Highlighting}
\end{Shaded}

\includegraphics{bacs_hw9_files/figure-latex/unnamed-chunk-14-1.pdf}

\hypertarget{b-2}{%
\subsection{3b}\label{b-2}}

\textbf{Question}

Do cars from different origins appear to have different weight vs.~mpg
relationships?

\textbf{Answer}

Yes, each of their data points seems clustered.

\end{document}
